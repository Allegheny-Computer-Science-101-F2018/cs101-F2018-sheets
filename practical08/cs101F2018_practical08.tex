\documentclass[11pt]{article}

% NOTE: The "Edit" sections are changed for each assignment

% Edit these commands for each assignment

\newcommand{\assignmentduedate}{November 19}
\newcommand{\assignmentassignedate}{November 16}
\newcommand{\assignmentnumber}{Eight}

\newcommand{\labyear}{2018}
\newcommand{\labday}{Friday}
\newcommand{\labdueday}{Wednesday}
\newcommand{\labtime}{1:30 pm}

\newcommand{\assigneddate}{Assigned: \labday, \assignmentassignedate, \labyear{} at \labtime{}}
\newcommand{\duedate}{Due: \labdueday, \assignmentduedate, \labyear{} at \labtime{}}

% Edit these commands to give the name to the main program

\newcommand{\mainprogram}{\lstinline{DoublyLinkedList}}
\newcommand{\mainprogramsource}{\lstinline{src/main/java/practicaleight/list/DoublyLinkedList.java}}

% Edit these commands to give the main program's output details

\newcommand{\mainprogramoutput}{four}

% Edit these commands to give the name to the test suite

\newcommand{\testprogram}{\lstinline{TestDoublyLinkedList}}
\newcommand{\testprogramsource}{\lstinline{src/test/java/practicaleight/TestDoublyLinkedList.java}}

% Edit this commands to describe key deliverables

\newcommand{\reflection}{\lstinline{writing/reflection.md}}

% Commands to describe key development tasks

% --> Running gatorgrader.sh
\newcommand{\gatorgraderstart}{\command{./gatorgrader.sh --start}}
\newcommand{\gatorgradercheck}{\command{./gatorgrader.sh --check}}

% --> Compiling and running and testing program with gradle
\newcommand{\gradlebuild}{\command{gradle build}}
\newcommand{\gradletest}{\command{gradle test}}
\newcommand{\gradlerun}{\command{gradle run}}

% Commands to describe key git tasks

% NOTE: Could be improved, problems due to nesting

\newcommand{\gitcommitfile}[1]{\command{git commit #1}}
\newcommand{\gitaddfile}[1]{\command{git add #1}}

\newcommand{\gitadd}{\command{git add}}
\newcommand{\gitcommit}{\command{git commit}}
\newcommand{\gitpush}{\command{git push}}
\newcommand{\gitpull}{\command{git pull}}

\newcommand{\gitcommitmainprogram}{\command{git commit src/main/java/practicaleight/list/DoublyLinkedList.java -m "Your
descriptive commit message"}}

% Use this when displaying a new command

\newcommand{\command}[1]{``\lstinline{#1}''}
\newcommand{\program}[1]{\lstinline{#1}}
\newcommand{\url}[1]{\lstinline{#1}}
\newcommand{\channel}[1]{\lstinline{#1}}
\newcommand{\option}[1]{``{#1}''}
\newcommand{\step}[1]{``{#1}''}

\usepackage{pifont}
\newcommand{\checkmark}{\ding{51}}
\newcommand{\naughtmark}{\ding{55}}

\usepackage{listings}
\lstset{
  basicstyle=\small\ttfamily,
  columns=flexible,
  breaklines=true
}

\usepackage{fancyhdr}

\usepackage[margin=1in]{geometry}
\usepackage{fancyhdr}

\pagestyle{fancy}

\fancyhf{}
\rhead{Computer Science 101}
\lhead{Practical Assignment \assignmentnumber{}}
\rfoot{Page \thepage}
\lfoot{\duedate}

\usepackage{titlesec}
\titlespacing\section{0pt}{6pt plus 4pt minus 2pt}{4pt plus 2pt minus 2pt}

\newcommand{\labtitle}[1]
{
  \begin{center}
    \begin{center}
      \bf
      CMPSC 101\\Data Abstraction\\
      Fall 2018\\
      \medskip
    \end{center}
    \bf
    #1
  \end{center}
}

\begin{document}

\thispagestyle{empty}

\labtitle{Practical \assignmentnumber{} \\ \assigneddate{} \\ \duedate{}}

\section*{Objectives}

To continue to practice using GitHub to access the files for an assignment. You
will complete a programming project using the Java source code given by the
course instructor, ultimately implementing and testing several methods provided
by the \mainprogram{}. Specifically, you will create and test a method that
strategically uses the \mainprogram's node structure to efficiently implement
the removal of the last node in the list. Also, you will implement and test both
a \program{main} method and a comprehensive method for ``equivalence testing''
with the \mainprogram{}. Finally, you will continue to learn how to implement
and test a Java program, practicing the use of an automated grading tool to
assess your progress towards correctly completing the project.

\section*{Suggestions for Success}

\begin{itemize}
  \setlength{\itemsep}{0pt}

\item {\bf Use the practical computers}. The computers in this practical feature specialized software for completing
  this course's practical and practical assignments. If it is necessary for you to work on a different machine, be sure
  to regularly transfer your work to a practical machine so that you can check its correctness. If you cannot use a
  practical computer and you need help with the configuration of your own laptop, then please carefully explain its
  setup to a teaching assistant or the course instructor when you are asking questions.

\item {\bf Follow each step carefully}. Slowly read each sentence in the assignment sheet, making sure that you
  precisely follow each instruction. Take notes about each step that you attempt, recording your questions and ideas and
  the challenges that you faced. If you are stuck, then please tell a teaching assistant or instructor what assignment
  step you recently completed.

\item {\bf Regularly ask and answer questions}. Please log into Slack at the start of a practical or practical session
  and then join the appropriate channel. If you have a question about one of the steps in an assignment, then you can
  post it to the designated channel. Or, you can ask a student sitting next to you or talk with a teaching assistant or
  the course instructor.

\item {\bf Store your files in GitHub}. As in previous course assignments, you will be responsible for storing
  all of your files (e.g., Java source code and Markdown-based writing) in a Git repository using GitHub Classroom.
  Please verify that you have saved your source code in your Git repository by using \command{git status} to ensure that
  everything is updated. You can see if your assignment submission meets the established correctness requirements by
  using the provided checking tools on your local computer and in checking the commits in GitHub.

\item {\bf Keep all of your files}. Don't delete your programs, output files, and written reports after you submit them
  through GitHub; you will need them again when you study for the quizzes and examinations and work on the other
  practical, practical, and final project assignments.

\item {\bf Back up your files regularly}. All of your files are regularly backed-up to the servers in the Department of
  Computer Science and, if you commit your files regularly, stored on GitHub. However, you may want to use a flash
  drive, Google Drive, or your favorite backup method to keep an extra copy of your files on reserve. In the event of
  any type of system failure, you are responsible for ensuring that you have access to a recent backup copy of all your
  files.

\item {\bf Explore teamwork and technologies}. While certain aspects of the practical assignments will be challenging
  for you, each part is designed to give you the opportunity to learn both fundamental concepts in the field of computer
  science and explore advanced technologies that are commonly employed at a wide variety of companies. To explore and
  develop new ideas, you should regularly communicate with your team and/or the teaching assistants and tutors.

\item {\bf Hone your Technical writing skills}. Computer science assignments require to you write technical
  documentation and descriptions of your experiences when completing each task. Take extra care to ensure that your
  writing is interesting and both grammatically and technically correct, remembering that computer scientists must
  effectively communicate and collaborate with their team members and the tutors, teaching assistants, and course
  instructor.

\item {\bf Review the Honor Code on the syllabus}. While you may discuss your assignments with others, copying source
  code or writing is a violation of Allegheny College's Honor Code.

\end{itemize}

\vspace*{-.2in}

\section*{Reading Assignment}

If you have not done so already, please read all of the relevant ``GitHub
Guides'', available at \url{https://guides.github.com/}, that explain how to use
many of the features that GitHub provides. In particular, please make sure that
you have read guides such as ``Mastering Markdown'' and ``Documenting Your
Projects on GitHub''; each of them will help you to understand how to use both
GitHub and GitHub Classroom. To do well on this practical assignment, you
should also read Section 1.5.2 in the textbook, paying particularly close
attention to the material about iteration constructs. You should also review the
content about lists and equivalence testing in Sections 3.3 through 3.6. Please
see the course instructor if you have questions about these reading assignments.

\section*{Accessing the Practical Assignment on GitHub}

To access the practical assignment, you should go into the
\channel{\#announcements} channel in our Slack team and find the announcement
that provides a link for it. Copy this link and paste it into a web browser.
Now, you should accept the assignment and see that GitHub Classroom created a
new GitHub repository for you to access the assignment's starting materials and
to store the completed version of your assignment. Specifically, to access your
new GitHub repository for this assignment, please click the green ``Accept''
button and then click the link that is prefaced with the label ``Your assignment
has been created here''. If you accepted the assignment and correctly followed
these steps, you should have created a GitHub repository with a name like
``Allegheny-Computer-Science-101-Fall-2018/computer-science-101-fall-2018-practical-8-gkapfham''.

% Unless you provide the instructor with documentation of the extenuating
% circumstances that you are facing, not accepting the assignment means that you
% automatically receive a failing grade for it.

Before you move to the next step of this assignment, please make sure that you
read all of the content on the web site for your new GitHub repository, paying
close attention to the technical details about the commands that you will type
and the output that your program must produce. Now you are ready to download the
starting materials to your practical computer. Click the ``Clone or download''
button and, after ensuring that you have selected ``Clone with SSH'', please
copy this command to your clipboard. To enter into your course directory you
should now type \command{cd cs101S2018}. By typing \command{git clone} in your
terminal and then pasting in the string that you copied from the GitHub site you
will download all of the code for this assignment. For instance, if the course
instructor ran the \command{git clone} command in the terminal, it would look
like:

\begin{lstlisting}
  git clone git@github.com:Allegheny-Computer-Science-101-S2018/computer-science-101-fall-2018-practical-7-gkapfham.git
\end{lstlisting}

After this command finishes, you can use \command{cd} to change into the new
directory. If you want to \step{go back} one directory from your current
location, then you can type the command \command{cd ..}. Please continue to use
the \command{cd} and \command{ls} commands to explore the files that you
automatically downloaded from GitHub. What files and directories do you see?
What do you think is their purpose? Spend some time exploring, sharing your
discoveries with a neighbor and a \mbox{teaching assistant}. Specifically, each
student should ensure that they can draw a technical diagram of a \mainprogram{}
and that they fully understand the purpose and behavior of every test case in
the \program{TestDoublyLinkedList}.

\section*{Implementation and Testing of a Doubly Linked Data Structure}

\begin{figure}[t]
  \centering
  \begin{verbatim}

    DoublyLinkedList Before removeLast:
    (0, 1, 2, 3, 4, 5, 6, 7, 8, 9)

    DoublyLinkedList After removeLast:
    (0, 1, 2, 3, 4, 5, 6, 7, 8)
  \end{verbatim}

  \vspace*{-.35in}
  \caption{The Expected Output of the \mainprogram{} Program.}~\label{fig:output}
  \vspace*{-.1in}
\end{figure}

\begin{figure}[t]
  \centering
  \begin{verbatim}

    public E removeLast() {
      if (isEmpty()) {
        return null;
      }
      return remove(trailer.getPrev());
    }

  \end{verbatim}

  \vspace*{-.6in}
  \caption{The Implementation of the \program{removeLast} Method.}~\label{fig:code}
  \vspace*{-.25in}
\end{figure}

This practical assignment invites you to implement a \mainprogram{} that can
produce the output given in Figure~\ref{fig:output}. Specifically, you must add
the source code provided in Figure~\ref{fig:code} to your implementation of the
\mainprogram{}. Is this an efficient implementation of this method? What is the
worst-case time complexity of this method? How can you tell? Now, you are also
responsible for adding source code to the \program{main} method in
\mainprogram{} so that it can produce the required output. The final step for
this practical assignment invites you to implement an \program{equals} method
that performs ``equivalence testing'' for two instances of \mainprogram{}. To
learn more about equivalence testing, please review the content in Section 3.5
of the textbook. Taking into consideration that the \mainprogram{} employs
header and trailer sentinels, your \program{equals} method should determine if
two instances contain the same data values. For inspiration, you can see a full
example of an \program{equals} method for the \program{SinglyLinkedList} by
looking at the source code for the most recent practical assignment. Make sure
that your \program{equals} does not call a method on a \program{null} object!

% Remember, if you want to \step{build} your program you can type the command
% \gradlebuild{} in your terminal, thereby causing the Java compiler to check your
% program for errors and get it ready to run. If you notice that some of the test
% cases do not pass, please improve your implementation until all of the tests
% pass and your program's output looks similar to that which is provided in
% Figure~\ref{fig:output}.

\section*{Checking the Correctness of Your Program}

As verified by the Checkstyle tool, the source code for the \mainprogram{} and
all of the other Java files must adhere to all of the requirements in the Google
Java Style Guide available at
\url{https://google.github.io/styleguide/javaguide.html}. Instead of requiring
you to manually check that your deliverables adhere to these industry-accepted
standards, GatorGrader makes it easy for you to automatically check if your
submission meets the correctness requirements. For instance, GatorGrader will
run your tests and check to ensure that certain files, like
\mainprogramsource{}, contain the right number of \program{println} statements.

To get started with the use of GatorGrader, type the command
\gatorgraderstart{} in your terminal window. Once this step completes you can
type \gatorgradercheck{}. If your work does not meet all of the assignment's
requirements, then you will see the following output in your terminal:
\command{Overall, are there any mistakes in the assignment? Yes}. If you do
have mistakes in your assignment, then you will need to review GatorGrader's
output, find the mistake, and try to fix it. Once your program is building
correctly, fulfilling at least some of the assignment's requirements, you
should transfer your files to GitHub using the \gitcommit{} and \gitpush{}
commands. For example, if you want to signal that the \mainprogramsource{} file
has been changed and is ready for transfer to GitHub you would first type
\gitcommitmainprogram{} in your terminal, followed by typing \gitpush{} and
checking to see that the transfer to GitHub is successful. If you notice that
your code transfer did not work, then please try to determine why, asking the
instructor for assistance, if necessary.

After the course instructor enables \step{continuous integration} with a system
called Travis CI, when you use the \gitpush{} command to transfer your source
code to your GitHub repository, Travis CI will initialize a \step{build} of your
assignment, checking to see if it meets all of the requirements. If both your
source code and documentation meet all of the established requirements, then you
will see a green \checkmark{} in the listing of commits in GitHub after awhile.
If your submission does not meet the requirements, a red \naughtmark{} will
appear instead. The instructor will reduce a student's grade for this assignment
if the red \naughtmark{} appears on the last commit in GitHub immediately before
the assignment's due date. Yet, if the green \checkmark{} appears on the last
commit in your GitHub repository, then you satisfied all of the main checks,
thereby allowing the course instructor to evaluate other aspects of your source
code and writing, as further described in the \step{Evaluation} section of this
assignment sheet. Unless you provide the instructor with documentation of the
extenuating circumstances that you are facing, no late work will be considered
towards your grade for this practical assignment.

\section*{Summary of the Required Deliverables}

\noindent Students do not need to submit printed source code or technical
writing for any assignment in this course. Instead, this assignment invites you
to submit, using GitHub, the following deliverables.

\vspace*{-.1in}

\begin{enumerate}

  \setlength{\itemsep}{0in}

\item A properly documented and correct version of all the Java source files
  (i.e., the \mainprogram{} and its JUnit test suite) that meets the set
  requirements and produces the desired output.

\end{enumerate}

\vspace*{-.2in}

\section*{Evaluation of Your Practical Assignment}

Using a report that the instructor shares with you through the commit log in
GitHub, you will privately received a grade on this assignment and feedback on
your submitted deliverables. Your grade for the assignment will be a function of
the whether or not it was submitted in a timely fashion and if your program
received a green \checkmark{} indicating that it met all of the requirements.
Other factors will also influence your final grade on the assignment. In
addition to studying the efficiency and effectiveness of your Java source code,
the instructor will also evaluate the accuracy of the technical writing in your
source code's comments. If your submission receives a red \naughtmark{}, the
instructor will reduce your grade for the assignment while still considering the
regularity with which you committed to your GitHub repository and the overall
quality of your partially completed work. Please see the instructor if you have
questions about the evaluation of this practical assignment.

\section*{Adhering to the Honor Code}

In adherence to the Honor Code, students should complete this assignment on an
individual basis. While it is appropriate for students in this class to have
high-level conversations about the assignment, it is necessary to distinguish
carefully between the student who discusses the principles underlying a problem
with others and the student who produces assignments that are identical to, or
merely variations on, someone else's work. Deliverables (e.g., Java source code
or Markdown-based technical writing) that are nearly identical to the work of
others will be taken as evidence of violating the \mbox{Honor Code}. Please see
the course instructor if you have questions about this policy.

\end{document}
